\section{University’s Plagiarism Statement}\label{s:plagiarism}

This statement is reviewed annually and the definitive statement is in the
\href{http://www.gla.ac.uk/services/senateoffice/policies/calendar/}{University
Calendar}. The University’s degrees and other academic awards are given in
recognition of a student’s personal achievement. All work submitted by students
for assessment is accepted on the understanding that it is the student’s own
effort.

Plagiarism is defined as the submission or presentation of work, in any form,
which is not one’s own, without acknowledgement of the sources. Plagiarism
includes inappropriate collaboration with others. Special cases of plagiarism
can arise from a student using his or her own previous work (termed
auto-plagiarism or self-plagiarism). Autoplagiarism includes using work that has
already been submitted for assessment at this University or for any other
academic award.

The incorporation of material without formal and proper acknowledgement (even
with no deliberate intent to cheat) can constitute plagiarism. Work may be
considered to be plagiarised if it consists of:

\begin{itemize}
    \item a direct quotation
    \item a close paraphrase
    \item an unacknowledged summary of a source
    \item direct copying or transcription
\end{itemize}

With regard to essays, reports and dissertations, the rule is: if information or
ideas are obtained from any source, that source must be acknowledged according
to the appropriate convention in that discipline; and any direct quotation must
be placed in quotation marks and the source cited immediately. Any failure to
acknowledge adequately or to cite properly other sources in submitted work is
plagiarism. Under examination conditions, material learnt by rote or close
paraphrase will be expected to follow the usual rules of reference citation
otherwise it will be considered as plagiarism. Departments should provide
guidance on other appropriate use of references in examination conditions.

Plagiarism is considered to be an act of fraudulence and an offence against
University discipline. Alleged plagiarism, at whatever stage of a student’s
studies, whether before or after graduation, will be investigated and dealt with
appropriately by the University.

The University reserves the right to use plagiarism detection systems, which may
be externally based, in the interests of improving academic standards when
assessing student work.