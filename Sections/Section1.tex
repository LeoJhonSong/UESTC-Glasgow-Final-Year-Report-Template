\section{Introduction}

Whew – I’ve finished the work, now I only have to write the report. Sound
familiar? Many students put off writing until too late and don’t leave enough
time to make a good job of the report. This is a serious mistake because the
report typically accounts for a large fraction of the assessment. Start early
because most people find report-writing difficult. Think about the report
throughout your project, keep track of references and collect material to
illustrate the report.

\textbf{The report contributes a large fraction of the overall assessment for
    the project.} It is depressingly common for students to perform excellent
technical work but submit a weak report, which pulls their overall result down.
Please apportion your time appropriately.

\subsection{What are the important parts of a report?}

It’s worth thinking of the sort of report that you might submit to your managers
in your future career. The technical content of the report can be broken into
two parts.

\begin{itemize}
    \item a description of the work that you did and the results that you
          obtained
    \item your analysis of the results and what you learned from them – the
          conclusions.
\end{itemize}

It is likely that your managers will open the report at the conclusions and
perhaps look back at the analysis if they want to see the evidence for a
particular point. It is most unlikely that they will read the earlier sections
at all. The report that you are writing now is an academic document and we will
read it all but the same point applies: the analysis and conclusions are the
most important sections. These are where you can show your understanding and
insight and they therefore make a major impact on your final grade. Make sure
that you spend enough time on them.

\subsection{Mechanical aspects}

The length of the body of the report is specified in the instructions for the
project. Extra material may be provided in appendices but this material is for
reference only: you cannot assume that the reader will study it. In other words,
do not put vital points in an appendix.

You probably think that the report is too short but this is deliberate. Most
reports are submitted to busy managers, who do not have time to read lengthy
documents. It is important to learn how to pick out the vital points and write a
concise report with maximum impact.

Reports should be word-processed and firmly bound. Use A4 paper and a clear
typeface such as 12-point Times\footnote{Cambria is better if you use a lot of
    mathematics in Word because it is used automatically for equations.}, number the
pages and leave margins of at least 25 mm all round. Follow the layout of this
document.

\begin{itemize}
    \item The front cover shows the title of the project with your name(s) and
          matriculation number(s).
    \item The abstract goes on the next page. It should be about 100–250 words
          and gives a brief summary of the report including the background and
          aims of the project, the principal results and conclusions.
    \item You may wish to include a page with acknowledgements next.
    \item The following page has the table of contents. There is no need for
          lists of figures and tables.
    \item The body of the report should be divided into numbered sections, each
          starting on a new page. Figures (diagrams, plots or photographs) and
          tables need captions and should be numbered.
    \item References follow the body of the report, again starting on a new
          page.
    \item Any appendices should again start on new pages. Number them
          alphabetically (Appendix A and so on).
\end{itemize}

\subsection{What goes into the introduction?}

Explain the background to the project and the reasons why your particular piece
of work was considered worthwhile. This usually includes a literature review to
show how your project builds on established knowledge. A worthwhile review must
be critical, meaning that you assess the previous publications to show where
more work is needed. A simple summary of previous work is of little value
because the reader can get this by scanning the references. Do not quote from
advertising material and do not make the literature review too long, definitely
less than 25\% of the report.

The introduction leads to the aims of the project: what you are trying to
achieve. Be specific.