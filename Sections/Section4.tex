\section{Conclusions and further work}

\subsection{Conclusions}

Every report must have conclusions, built on the discussion. This section
includes a summary of the main achievements of the work but is more than that.
The noun ‘conclusion’ can be defined as \textquote{a judgement or decision
reached by reasoning} \autocite{oxford} so you should highlight what has been
learnt as a result of your project. What did your project achieve – what is the
big picture that the reader should take away?

The Conclusions should include an assessment of the outcome against the original
aims of the project. If you were unable to fulfil some aims, explain why. It is
also useful to review the project plan (which should be included with the
report). Did some tasks prove to be unexpectedly difficult, for instance?

\subsection{Suggestions for further work}

Almost every project leaves you with ideas for the future. Research typically
answers some questions while opening new ones; by the end of a design-and-build
project you may have thought of a superior approach. Examiners are impressed by
intelligent suggestions for further work because they show that you really
understand the project. It is often a sign of strength to identify the weak
points and suggest solutions for them.

The \textit{Analysis, Discussion and Conclusions sections are important} –
perhaps the most important parts of the report. Many students do not spend
enough time on these sections but this is where you can display your
understanding of the project, your insight into the technical analysis and your
appreciation of the impact of the results.